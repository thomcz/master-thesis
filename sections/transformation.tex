%% LaTeX2e class for student theses
%% sections/evaluation.tex
%% 
%% Karlsruhe Institute of Technology
%% Institute for Program Structures and Data Organization
%% Chair for Software Design and Quality (SDQ)
%%
%% Dr.-Ing. Erik Burger
%% burger@kit.edu
%%
%% Version 1.3.3, 2018-04-17

\chapter{Transformation}
\label{ch:transformation}
Da Palladio bereits einen Modellelemente bereitstellt um eventbasierte Komunikation abzubilden, ist die Idee, diese Elemente wiederzuverwenden. Das Problem ist, dass beim Verwenden von Events die Modeltransformation eine Komponentenkette einwebt die den Anforderungen einer MOM, wie sie in dieser Masterarbeit beschrieben wurde, nicht genügt. Deshalb soll diese Modelltransformation ersetzt werden. Der Vorteil dabei ist, dass die Event-Elemente wiederverwendet werden können und dem Benutzer Modellierungsaufwand erspart bleibt. Wie im vorherigen Kapitel besprochen, kann eine große MOM-Architektur viele Warteschlangen und Exchange-Komponenten, die miteinander agieren, hervorbringen. Im Folgenden wird beschrieben, wie eine solche Modelltransformation aussehen kann.
%Idee des Event Mechanismus: man moechte konkret ueber nachrichten/events reden und nicht abstrakt ueber interfaces. Event Transformation liefert am Ende Modell mit Interfaces usw. \\

\section{Beschreibung}
Sender Komponente hat:\\
- SourceRole mit EventGroup (EG)\\
- Seff mit EmitAction\\
-- EventType (ET) der zu EG passt\\
-- SourceRole EG (SoR) \\

Empfaenger Komponente hat:\\
- SinkRole (SiR) (EG wie Sender)\\
- Seff (EventHandler, d.h. als Described Service ist ET eingetragen) hat ET vom Sender


Transformation\\
Repository\\
- Fuege MOM Komponente mit provided IMOM ein + Queue
(MOM entweder nur per Transformation generieren oder auch wie in Eventtransformation ueber Namen)

Sender\\
- Sender erstellt RequiredRole fuer IMOM "senderRequiresMOM"\\
- Die EmitAction des Sender Seffs wird mit ExternalCall ersetzt\\
-- CalledService: IMOM.distribute\\
-- RoleExternalService: "senderRequiresMOM"\\
-- uebergebe UsageVariable mit Type: EventType\\

Receiver\\
(- erstelle IReceiver mit Operation recvMsg (Global)\\
- Receiver provided IReceiver\\
- erstelle in SEFF GuardedBranches\\
-- fuer jede sourceRole erstelle GuardedBranch, der ueber UsageVariable und ET unterschieden wird)\\

- Fuer jede SourceRole erstelle IReceiverSourceRole Interface, dass provided wird\\
- EventHandler Seff aendert Described Seff zu jeweilige IReceiverSourceRole.recvMsg\\
- Receiver erstellt RequiredRole fuer IMOM "recvRequiresMOM"\\
-- uebergebe UsageVariable mit Type: EventType\\

MOM\\
- fuer jeden EventType erstelle Queue\\
-- passiveRessource \\
-- Aquire/Release mechanismus\\
-- provided IQueue\\

- fuer jeden EventType erstelle requiredRole fuer obige provided Queues\\
- in receive Seff\\
-- branch Action die entscheided welche Queue genommen wird (UsageVariable == EventType) zum aquire\\

- in distribute Seff\\
-- branch Action die entscheided welche Queue genommen wird (UsageVariable == EventType) zum release\\

System\\
- laufe durch Repository und deploye alles :D\\
- fuer jeden IReceiver provide einen SystemProvide\\
- finde zu jedem Source einen Sink und direkte Verbindung/schalte EventChannel dazwischen\\

Usage\\
- fuer jeden SystemProvide ein UsageScenario, das Nachrichten empfaengt\\

ResourceEnv\\
- Middleware Container generieren oder vom User modellieren lassen und per Namenskonvention deployen\\

Allocation\\
- alle Queues + IMOM (+ EventChannels) werden auf gleichen ResouceContainer deployed\\