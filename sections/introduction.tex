%% LaTeX2e class for student theses
%% sections/content.tex
%% 
%% Karlsruhe Institute of Technology
%% Institute for Program Structures and Data Organization
%% Chair for Software Design and Quality (SDQ)
%%
%% Dr.-Ing. Erik Burger
%% burger@kit.edu
%% burger@kit.edu
%%
%% Version 1.3.3, 2018-04-17

\chapter{Einleitung}
\label{ch:Introduction}
Moderne Softwaresysteme werden über immer größere Entfernungen verteilt, laufen in unterschiedlichen Umgebungen, auf verschiedener Hardware und Betriebssystemen und sind noch dazu in verschiedenen Programmiersprachen geschrieben. Gleichzeitig werden hohe Anforderungen an Performance, Verfügbarkeit und Sicherheit gestellt. Dies führt dazu, dass auch die Anforderungen an ihre Kommunikationsinfrastruktur steigen. \par
Mithilfe von nachrichtenbasierter Middleware (MOM) lassen sich diese Anforderungen erfüllen. Ein Benutzer einer MOM kann sowohl Nachrichten empfangen, als auch senden. Dazu muss er nur mit der MOM verbunden sein. Er muss keine Informationen über die anderen Benutzer haben. Die MOM fügt eine Zwischenschicht zwischen den Sendern und Empfängern ein und ermöglicht dadurch Entkopplung und gleichzeitig einen asynchronen Nachrichtenaustausch, sodass die einzelnen Benutzer nicht blockieren müssen, wenn sie eine Nachricht senden oder empfangen wollen. \par
\emph{ActiveMQ}, \emph{Kafka}, \emph{RabbitMQ}, \emph{Amazon MQ} oder \emph{ZeroMQ} sind nur ein Paar Beispiele von MOMs. Es gibt eine Vielzahl von verschiedenen MOMs, die jeweils unterschiedliche Ziele oder Schwerpunkte haben. \emph{Kafka} legt zum Beispiel mehr Wert auf Performance als auf Verfügbarkeit \cite{kafka}, während \emph{RabbitMQ} eine allseitig einsetzbare MOM ist \cite{rabbitmq}. \par
Um aus dieser Vielzahl an MOMs eine MOM für ein bestimmtes Softwaresystem auszuwählen ist bereits Expertenwissen notwendig. Grund dafür ist neben der Vielzahl auch die hohe Konfigurierbarkeit der einzelnen MOMs. Diese ist gegeben, damit diese einen weiten Anwendungsbereich abdecken kann. Der jeweilige Benutzer kann dann auf spezielle Anforderungen seines Systems reagieren und die MOM dementsprechend anpassen. Eine mögliche Konfiguration könnte zum Beispiel dazu führen, dass die MOM weniger performant ist, aber dafür die Verfügbarkeit verbessert wird. Eine MOM so zu konfigurieren, dass sie den Wünschen des Benutzer entspricht ist nicht trivial und benötigt Expertenwissen. 

\section{Ziel}
Das Ziel dieser Masterarbeit ist es, den Softwarearchitekten bei der Wahl einer MOM, für ein bestimmtes  Softwaresystem, bereits in der Designphase, zu unterstützen. Um dies zu ermöglichen, soll es für den Softwarearchitekten möglich sein einen wiederverwendbaren MOM-Modellbaustein auszuwählen und damit eine Softwarearchitektur, mithilfe einer Performance-Analyse, zu untersuchen. Somit ist bereits auf Modellebene die Auswirkung der Entscheidung zu sehen und der Architekt kann früh darauf reagieren. Um dies zu erreichen, soll im Rahmen dieser Masterarbeit eine Methode entwickelt werden, wie aus einer MOM ein wiederverwendbarer Modellbaustein wird. Dazu wird zunächst eine MOM mit einem Benchmark ausgemessen und im Anschluss modelliert. Mithilfe der Messergebnisse kann die Modellierung kalibriert werden, um im Anschluss eine Performance-Analyse zu ermöglichen. Außerdem wurde eine Modeltransformation konzipiert, um bereits existierende Modell-Elemente, zum Abbilden eventbasierter-Kommunikation, wiederzuverwenden. Mithilfe eines Evaluierungssystem soll schließlich überprüft werden, ob mithilfe des Prozesses das Ziel der Masterarbeit erreicht werden konnte.

\section{Aufbau}
Im Folgenden wird der Aufbau der Masterarbeit beschrieben. In \autoref{ch:Grundlagen} werden Grundlagen erläutert, die für die weitere Arbeit benötigt werden. In \autoref{ch:Verwandte} wird ein Überblick über themenverwandte Arbeiten gegeben. Sie werden kurz zusammengefasst und diskutiert, in
wie weit sie für die Masterarbeit relevant sind, bzw. wie sich die Masterarbeit abgrenzt. \autoref{ch:mom} wählt und untersucht eine konkrete MOM. Die Modellierung einer MOM wird anschließend in \autoref{ch:modellierung} beschrieben. Um bestehende Elemente wiederverwenden zu können wird in \autoref{ch:transformation} eine Modelltransformation beschrieben. In \autoref{ch:Evaluation} wird die Modellierung mithilfe eines Fallbeispiels evaluiert. Schließlich wird in \autoref{ch:zusammenfassung} die Masterarbeit zusammengefasst und ein Ausblick für zukünftige Arbeiten gegeben.
%Wieso werden mqs verwendet? \\
%- entkoplung \\
%- skallierbarkeit (horizontal/vertikal) \\
%- async Kommunikation \\
%% batching (mehrere Eintraege auf einmal einfuegen)\\
%- wiederverwendbarkeit\\
%- arbeit aufteilen \\
%- bestimmte Garantien die MQ Liefert (reihenfolge, garantierte Lieferung) \\
%- lastspitzen puffern \\
%- gegen ausfaelle schutzen (Redundanzen) \\


%Problem: \\
%-	Keine Performanz Vorhersage für MQs \\
%-	Auswirkungen von Konfigurationen unbekannt \\
%Idee: wollen bei auswahl von MQ System unterstuetzen \\
%- sowohl bei der Plannung als auch in der Evolution des Systems \\
%- Nicht den Entwickler des MQ Systems, aber wie Würde das aussehen, wenn doch? (Abgrenzung) \\
%Vorteil:\\
%- MQ Abwaegung auf Architekturebene\\
%Aktion:\\
%- experiment \\
%%- modellierung \\
%- evaluierung 
