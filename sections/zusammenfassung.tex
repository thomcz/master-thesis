%% LaTeX2e class for student theses
%% sections/evaluation.tex
%% 
%% Karlsruhe Institute of Technology
%% Institute for Program Structures and Data Organization
%% Chair for Software Design and Quality (SDQ)
%%
%% Dr.-Ing. Erik Burger
%% burger@kit.edu
%%
%% Version 1.3.3, 2018-04-17

\chapter{Zusammenfassung}
\label{ch:zusammenfassung}


%1. Zusammenfassung: Was wurde in dieser Arbeit gemacht? Was waren die Schlüsselergebnisse? Diesmal jedoch zusammengefasst nochmals für einen Leser, der die vorherigen Seiten mit allen Details bereits gelesen hat.
Prozess um MOM auszumessen, modellieren, kalibrieren, transformieren und evaluieren.
%2. Adressaten der Verbesserung: Wem nützen die Verbesserungen/Beiträge der Arbeit? Inwieweit wird die Software-Technik durch die Arbeit verbessert?
Softwarearchitekt kann Auswirkung einer MOM an seiner Software-Architektur sehen. 
%3. Aufbauende/Zukünftige Arbeiten: Welche nächsten Schritte sind geplant (erst kurzfristige, dann längerfristige)? Welche möglichen Lösungsansätze für noch bestehende Probleme sind denkbar? Wie könnten Folgearbeiten aussehen?
Ausblick\\
Nachrichtenanzahl und Nachrichtengröße betrachten bei der Ausmessung \\
Messe Zeit von Sendenr und Empfänger getrennt und pflege es in Modell ein, momentan ist alles in Sender eingetragen \\
weitere effekte \\
batching \\
wartschlange als tatsächliches PCM Element nicht über passive Ressource (kann nachricht durch system tracken)